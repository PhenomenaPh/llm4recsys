\section*{Заключение}
\addcontentsline{toc}{section}{Заключение}

В рамках данного исследования была разработана и экспериментально апробирована методология применения больших языковых моделей для решения задачи персонализированных рекомендаций контента. Проведенная работа включала несколько ключевых этапов: от предобработки данных и создания семантических представлений до дообучения языковой модели и оценки её эффективности.

Основные результаты исследования:

\begin{itemize}
    \item \textbf{Успешная реализация гибридного подхода}: Разработанная двухуровневая архитектура, объединяющая иерархическую кластеризацию контента на основе текстовых эмбеддингов и дообученную языковую модель для генерации переходов между кластерами, продемонстрировала высокую эффективность.
    
    \item \textbf{Высокие показатели качества}: Дообученная модель Llama-3.1-8B с архитектурой LoRA достигла значения Article Match Rate на уровне 99.6\%, что свидетельствует о способности генерировать семантически корректные и отличающиеся от предыдущих рекомендации.
    
    \item \textbf{Эффективность параметрически-экономичного обучения}: Применение LoRA позволило сократить количество обучаемых параметров до 2.05\% от общего объема модели при сохранении высокого качества дообучения.
    
    \item \textbf{Решение проблемы замкнутости}: Предложенный подход успешно адресует классическую проблему замкнутости рекомендательных систем, позволяя пользователям исследовать новые области интересов за пределами их привычных предпочтений.
    
    \item \textbf{Масштабируемость решения}: Использование кластеризации контента и работа на уровне семантических групп, а не отдельных объектов, обеспечивает возможность применения подхода к каталогам различного размера.
\end{itemize}

Проведенный анализ показал, что интеграция больших языковых моделей в рекомендательные системы открывает новые возможности для создания более интеллектуальных и адаптивных решений. LLM продемонстрировали способность к пониманию сложных семантических связей между различными типами контента и генерации релевантных рекомендаций на основе ограниченного контекста.

Развитие больших языковых моделей в задачах рекомендаций выглядит крайне многообещающим. Возможности современных LLM в области понимания естественного языка, способности к обобщению и контекстному рассуждению создают предпосылки для революционных изменений в области персонализации контента. Ожидается, что дальнейшее совершенствование архитектур языковых моделей, методов их обучения и интеграции с традиционными рекомендательными подходами приведет к созданию нового поколения рекомендательных систем, способных обеспечить более глубокое понимание пользовательских потребностей и предоставить качественно новый уровень персонализированного опыта.

Результаты данной работы вносят вклад в понимание практических аспектов применения LLM в индустриальных рекомендательных системах и могут служить основой для дальнейших исследований в этой динамично развивающейся области.

