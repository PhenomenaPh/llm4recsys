Рекомендательные системы (РС) стали неотъемлемой частью современных онлайн-платформ, помогая пользователям ориентироваться в огромных объемах информации и находить релевантный контент, продукты или услуги. Традиционные подходы к построению РС, такие как коллаборативная фильтрация и контент-ориентированные методы, достигли значительных успехов, однако они часто сталкиваются с проблемами разреженности данных, холодного старта и ограниченной способностью к пониманию семантических нюансов пользовательских предпочтений и описаний элементов.

В последнее время наблюдается значительный рост интереса к большим языковым моделям (Large Language Models, LLM), которые демонстрируют уникальные способности в понимании, генерации и обработке естественного языка. Это привело к открытию новых горизонтов для различных областей искусственного интеллекта, включая рекомендательные системы. Активно изучается возможность применения LLM для решения классических проблем РС и создания более продвинутых, персонализированных и интерактивных рекомендательных сервисов. Современные исследования выделяют два основных направления применения LLM в рекомендательных системах: дискриминативные модели (DLLM4Rec), ориентированные на классификацию и ранжирование, и генеративные модели (GLLM4Rec), способные создавать персонализированные рекомендации и объяснения в естественном языке \citep{wu2024surveylargelanguagemodels}. Кроме того, появляется интерес к агентным системам, где LLM выступают в роли интеллектуальных агентов, способных к диалоговому взаимодействию и стратегическому планированию рекомендаций \citep{peng2025surveyllmpoweredagentsrecommender}.

Однако, несмотря на значительный потенциал, интеграция LLM в рекомендательные системы сопряжена с рядом вызовов. К ним относятся высокая чувствительность к формулировке запросов, возможность генерации некорректных или несуществующих рекомендаций, а также значительные вычислительные затраты при внедрении в реальные приложения. Тем не менее, активные исследования в этой области продолжаются, предлагая новые архитектурные решения и методы адаптации больших языковых моделей для улучшения качества рекомендаций и повышения доверия пользователей \citep{vats2024exploringimpactlargelanguage}.


\underline{Цель работы:} разработка и оценка эффективности внедрения больших языковых моделей в области рекомендательных систем.

\underline{Задачи работы:}
\begin{itemize}
    \item Провести систематический обзор литературы по применению больших языковых моделей в рекомендательных системах
    \item Разработать методологию создания семантических представлений контента на основе текстовых эмбеддингов и иерархической кластеризации
    \item Реализовать двухуровневую архитектуру рекомендательной системы, использующую LLM для генерации переходов между кластерами интересов
    \item Адаптировать большую языковую модель с использованием LoRA-архитектуры для задачи предсказания пользовательских предпочтений
    \item Провести экспериментальную оценку эффективности предложенного подхода на датасете MTS Kion
    \item Проанализировать способность модели решать проблему замкнутости рекомендательных систем
\end{itemize}

\underline{Методы исследования:}
\begin{itemize}
    \item Анализ литературы и систематизация подходов к применению LLM в рекомендательных системах
    \item Предобработка данных и создание текстовых описаний контента с генерацией ключевых слов через LLM
    \item Создание семантических эмбеддингов с помощью модели SentenceTransformer (Jina-v3) и иерархическая кластеризация контента
    \item Извлечение характеризующих ключевых слов кластеров с использованием TF-IDF анализа
    \item Параметрически-эффективное дообучение языковой модели Llama-3.1-8B с применением LoRA-архитектуры
    \item Разработка специализированных метрик оценки: recall\_exact\_completion, avg\_jaccard\_to\_true, article\_match\_rate
    \item Количественный и качественный анализ результатов генерации рекомендаций
\end{itemize}

\underline{Объект исследования:} набор реальных данных из приложения МТС Kion по взаимодействиям пользователей с контентом за период 6 месяцев в 2021 году, включающий в себя факты просмотра контента пользователями, описания различных аттрибутов контента, а также описательные метрики пользователя\footnote{MTS Kion Implicit Contextualised Sequential Dataset for Movie Recommendation // URL: \url{https://github.com/irsafilo/KION_DATASET?tab=readme-ov-file}}.

\underline{Актуальность работы:} заключается в анализе и применении передовых подходов в задаче использования LLM для рекомендации контента, что позволяет повысить качество персонализации и решить классические проблемы рекомендательных систем. 